% Options for packages loaded elsewhere
\PassOptionsToPackage{unicode}{hyperref}
\PassOptionsToPackage{hyphens}{url}
%
\documentclass[
]{book}
\usepackage{amsmath,amssymb}
\usepackage{iftex}
\ifPDFTeX
  \usepackage[T1]{fontenc}
  \usepackage[utf8]{inputenc}
  \usepackage{textcomp} % provide euro and other symbols
\else % if luatex or xetex
  \usepackage{unicode-math} % this also loads fontspec
  \defaultfontfeatures{Scale=MatchLowercase}
  \defaultfontfeatures[\rmfamily]{Ligatures=TeX,Scale=1}
\fi
\usepackage{lmodern}
\ifPDFTeX\else
  % xetex/luatex font selection
\fi
% Use upquote if available, for straight quotes in verbatim environments
\IfFileExists{upquote.sty}{\usepackage{upquote}}{}
\IfFileExists{microtype.sty}{% use microtype if available
  \usepackage[]{microtype}
  \UseMicrotypeSet[protrusion]{basicmath} % disable protrusion for tt fonts
}{}
\makeatletter
\@ifundefined{KOMAClassName}{% if non-KOMA class
  \IfFileExists{parskip.sty}{%
    \usepackage{parskip}
  }{% else
    \setlength{\parindent}{0pt}
    \setlength{\parskip}{6pt plus 2pt minus 1pt}}
}{% if KOMA class
  \KOMAoptions{parskip=half}}
\makeatother
\usepackage{xcolor}
\usepackage{longtable,booktabs,array}
\usepackage{calc} % for calculating minipage widths
% Correct order of tables after \paragraph or \subparagraph
\usepackage{etoolbox}
\makeatletter
\patchcmd\longtable{\par}{\if@noskipsec\mbox{}\fi\par}{}{}
\makeatother
% Allow footnotes in longtable head/foot
\IfFileExists{footnotehyper.sty}{\usepackage{footnotehyper}}{\usepackage{footnote}}
\makesavenoteenv{longtable}
\usepackage{graphicx}
\makeatletter
\def\maxwidth{\ifdim\Gin@nat@width>\linewidth\linewidth\else\Gin@nat@width\fi}
\def\maxheight{\ifdim\Gin@nat@height>\textheight\textheight\else\Gin@nat@height\fi}
\makeatother
% Scale images if necessary, so that they will not overflow the page
% margins by default, and it is still possible to overwrite the defaults
% using explicit options in \includegraphics[width, height, ...]{}
\setkeys{Gin}{width=\maxwidth,height=\maxheight,keepaspectratio}
% Set default figure placement to htbp
\makeatletter
\def\fps@figure{htbp}
\makeatother
\setlength{\emergencystretch}{3em} % prevent overfull lines
\providecommand{\tightlist}{%
  \setlength{\itemsep}{0pt}\setlength{\parskip}{0pt}}
\setcounter{secnumdepth}{5}
\usepackage{booktabs}
\ifLuaTeX
  \usepackage{selnolig}  % disable illegal ligatures
\fi
\usepackage[]{natbib}
\bibliographystyle{apalike}
\IfFileExists{bookmark.sty}{\usepackage{bookmark}}{\usepackage{hyperref}}
\IfFileExists{xurl.sty}{\usepackage{xurl}}{} % add URL line breaks if available
\urlstyle{same}
\hypersetup{
  pdftitle={Study note template},
  pdfauthor={Jung Xue},
  hidelinks,
  pdfcreator={LaTeX via pandoc}}

\title{Study note template}
\author{Jung Xue}
\date{2023-09-14}

\begin{document}
\maketitle

{
\setcounter{tocdepth}{1}
\tableofcontents
}
\hypertarget{ch2}{%
\chapter{Chapter title}\label{ch2}}

\hypertarget{subsection}{%
\section{subsection}\label{subsection}}

\hypertarget{ch3}{%
\chapter{Chapter title}\label{ch3}}

\hypertarget{subsection-1}{%
\section{subsection}\label{subsection-1}}

\hypertarget{ch4}{%
\chapter{Chapter title}\label{ch4}}

\hypertarget{subsection-2}{%
\section{subsection}\label{subsection-2}}

\hypertarget{concluding-remarks}{%
\chapter{Concluding Remarks}\label{concluding-remarks}}

What did you learnt by the end of this session/course?

Take home message?

Add 3 questions to ponder.

\hypertarget{how-to-use-rbookdown}{%
\chapter*{How to use RBookDown}\label{how-to-use-rbookdown}}
\addcontentsline{toc}{chapter}{How to use RBookDown}

Firstly, you will have to read the \href{https://bookdown.org/yihui/bookdown/}{RBookDown Bible} by YiHui Xie

In essence, you write in a mixture of markdown (For basics), html (to extend on markdown) and latex language (mostly for equations) to create a simple Note.

You can customise your style and theme through your own CSS.

RMarkdown are mostly used to knit e-books(HTML), use TexStudio if you want a proper PDF, it is easier.

\textbf{Here are some useful tips to get started}

1: To add a chapter, just open a R file and save as \texttt{.RMD}. Use number 0 to 99 with a hyphen \texttt{-} to order the RMD files and maybe add a Chapter name so it is easier to select from \texttt{Files} window at bottom right of the R Studio.

2: Code chunks can generate graphical outputs, To insert pictures just use \texttt{include\_graphics} instead of \texttt{\textbackslash{}includegraphics\{\}} or \texttt{!{[}{]}()}. Width can be customised.

\begin{verbatim}
knitr::include_graphics(rep('images/knit-logo.png', 3))
\end{verbatim}

3: Use 1 grave accent ` to include the inline code, use 3 grave accent to include a chunk of code.

  \bibliography{references.bib}

\end{document}
